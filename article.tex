\documentclass{svproc}

\usepackage{url}
\usepackage{graphicx}
\usepackage[utf8]{inputenc}
\usepackage{float}
\usepackage{multicol}
\pagestyle{plain}

\def\UrlFont{\rmfamily}

\begin{document}
	\mainmatter
	\title{SOFTWARE DESIGN FOR A ROBOT TO ASSIST LAPAROSCOPIC SURGERY}
	%
	\titlerunning{Software Design} 
	
	\author{Dayanne Fernandes}
	
	\institute{Robotics and Automation Laboratory (LARA), \\
		University of Brasilia, Brazil\\
		\email{dayannefernandesc@gmail.com}}
	
	\maketitle              
	
	\begin{abstract}
		
		\keywords{Software Design, Robotics, Laparoscopic Surgery, CLARA, Julius, QT, C++}
	\end{abstract}
	
	\begin{multicols}{2}
		
	\section{Introduction}
	
	% laparoscopic surgery
	
	In the beginning of 1980s, laparoscopic surgery was migrated from diagnostic to a surgical procedure. Semm K and Muehe E are the medics that introduced this technique for a wide field of indications, e.g., appendectomy, cholecystectomy, reflux surgery, gastric surgery, urology \cite{pmid26713285}. 
	
	% robots to aid laparoscopic surgery
	Nowadays the use of robotic surgery system are becoming more common, its benefits are being studied all around the world. Robotic technology allows the surgeon to increase dexterity and the degree of maneuverability to perform complex tasks in a minimally invasive fashion way \cite{doi:10.3109/10929088.2010.541620}. 
	
	% current robots to aid laparoscopic surgery 
	
 	Laparoscopic surgery demands an operating surgeon to make the medical procedure and a camera driver to show the surgeon the location of the operative field. This structure come with lots of problems, e.g. conflicts about the optimum visualization, fatigue of the assistant who holds the camera \cite{pmid28643066}, inaccurate movements, image tremors.
 	
 	There are many projects to solve the problems about the fatigue during laparoscopic surgery, e.g., EndoAssist \cite{pmid25484949}, Vicky \cite{pmid17867952}. These solutions are very expansive to apply to Brazilian's public health system (\textit{SUS}). Aiming to develop a project with low costs and also help the operating surgeon to procedure the surgery, the project CLARA was created at \textit{LARA} (Robotics and Automation Laboratory, University of Brasilia).
 	
 	% clara project 
 	
 	\textit{CLARA}'s project was designed to help the operating surgeon to perform the procedure with full control of the tools through effective interfaces, voice and joystick linked with the laparoscopic grasper. The voice control was compared with foot pedal interface, and they concluded that voice control was more accurate and had the advantage of not requiring the surgeon to look away from the operative field \cite{pmid16844449}. 
 	
 	% clara software and other softwares to aid biomedical robots (ros)
 	
 	The software was designed them to be applied into public hospitals. Compatibility, modularity, fault-tolerance, security, usability, performance and maintainability were topics considered during the software modeling process. This paper shows a software designed to assist laparoscopic robot surgery with all the aspects cited above.
 	
	\section{CLARA Structure}
	
	% mechanical
	
	% hardware
	
	% software
	
	\section{Software Design}
	
	% libraries used
	
	Even thought there are other frameworks to help produce robotics projects, like ROS \cite{288}, the usability and the user experience was put as a priority to produce \textit{CLARA}'s software. The framework \textit{QT}\cite{qt} was used to develop the software, is a cross-platform framework, offers tools to pass messages (events) between modules in a secure and fast way and have a easy GUI (graphical user interface) module to use.

	% modules 
	
	\section{Conclusion}

	% Why the software design elaborated on CLARA project was important and efficient and effective 
	
	\section{Acknowledgements}
	
	CLARA project is funded by the Brazilian Ministry of Health (Term of Cooperation no 121/2013, Process 25000.169843/2013-83). 
		
	% ---- Bibliography ----
	\bibliographystyle{unsrt}
	\bibliography{article} 
	
	\end{multicols}
	
\end{document}
